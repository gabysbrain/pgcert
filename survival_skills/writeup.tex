% Options for packages loaded elsewhere
\PassOptionsToPackage{unicode}{hyperref}
\PassOptionsToPackage{hyphens}{url}
%
\documentclass[
]{article}
\usepackage{lmodern}
\usepackage{amssymb,amsmath}
\usepackage{ifxetex,ifluatex}
\ifnum 0\ifxetex 1\fi\ifluatex 1\fi=0 % if pdftex
  \usepackage[T1]{fontenc}
  \usepackage[utf8]{inputenc}
  \usepackage{textcomp} % provide euro and other symbols
\else % if luatex or xetex
  \usepackage{unicode-math}
  \defaultfontfeatures{Scale=MatchLowercase}
  \defaultfontfeatures[\rmfamily]{Ligatures=TeX,Scale=1}
\fi
% Use upquote if available, for straight quotes in verbatim environments
\IfFileExists{upquote.sty}{\usepackage{upquote}}{}
\IfFileExists{microtype.sty}{% use microtype if available
  \usepackage[]{microtype}
  \UseMicrotypeSet[protrusion]{basicmath} % disable protrusion for tt fonts
}{}
\makeatletter
\@ifundefined{KOMAClassName}{% if non-KOMA class
  \IfFileExists{parskip.sty}{%
    \usepackage{parskip}
  }{% else
    \setlength{\parindent}{0pt}
    \setlength{\parskip}{6pt plus 2pt minus 1pt}}
}{% if KOMA class
  \KOMAoptions{parskip=half}}
\makeatother
\usepackage{xcolor}
\IfFileExists{xurl.sty}{\usepackage{xurl}}{} % add URL line breaks if available
\IfFileExists{bookmark.sty}{\usepackage{bookmark}}{\usepackage{hyperref}}
\hypersetup{
  pdftitle={Reflections on survival skills},
  pdfauthor={Thomas Torsney-Weir},
  hidelinks,
  pdfcreator={LaTeX via pandoc}}
\urlstyle{same} % disable monospaced font for URLs
\setlength{\emergencystretch}{3em} % prevent overfull lines
\providecommand{\tightlist}{%
  \setlength{\itemsep}{0pt}\setlength{\parskip}{0pt}}
\setcounter{secnumdepth}{-\maxdimen} % remove section numbering

\usepackage{apacite}

\title{Reflections on survival skills}
\author{Thomas Torsney-Weir}
\date{}

\begin{document}
\maketitle

As an educator, it is important to be oneself. My personal style is
casual and interactive. I feel that it is my responsibility to keep
students' interest in the lectures. The lecture should be worthwhile for
them to attend despite distractions such as phones and sunny days.

I was interested in the survival skills patch because the stated goal is
``to raise participants' awareness of techniques and approaches.'' I
thought I would learn some practical tips for preparing lectures given a
limited time budget. I do enjoy teaching and learning about new ways to
teach, but I feel that I could be much more efficient with my time
before, during, and after my lectures. I found that while the content of
the patch itself was not directly transferable to my teaching, during
the patch I am encouraged to look further into solutions that may work
better for me.

I will reflect on this using Borton's model of reflection~\cite{Borton:1970}.
This breaks the reflective process down into three parts: what, so what, and
now what. The ``What'' phase describes the event or task that prompted the
reflection. ``So what'' is the analysis of the event. Finally, ``now what'' is
where one reflects on possible improvements or solutions. I found this model a
bit cumbersome. I usually think about things in terms of a two stage model:
``problem/solution.'' The solution part has pretty clear ties to the ``now
what'' stage of Borton's model. However, Borton's ``what'' and ``so what'' are
generally intermixed within the problem description. I am still not convinced
this needs to be extracted separately but it was an interesting exercise
nonetheless.

\hypertarget{what}{%
\section{What?}\label{what}}

Before my lectures involves how to plan efficiently. During the lecture
I feel that I go on many tangents that, while engaging, take away from
retention of knowledge. After the lecture, how do I encourage students
to practice lesson content? For example, I struggle with ways to
convince struggling students to come to office hours.

In general, I found the survival skills patch helpful with
recommendations to understand class size and student motivation. The
possibility of having a review quiz at the beginning of every lecture
and/or a review session at the end is an interesting idea and I'd like
to think of how I can incorporate it into my personal teaching style.
During the discussions, I found it very surprising how much quizzing
some lecturers do!

However, I feel that while the patch content was engaging during the
lecture, afterwards I struggle to come up with actionable ideas that I
can use in my lectures. I would have appreciated less breaking out into
small groups to discuss student issues and more discussion of practical
examples. I also thought that the planning and presentation preparation
recommendations were a little too general to incorporate directly into
my practice.

\hypertarget{so-what}{%
\section{So what?}\label{so-what}}

I've gained a lot of insight into other teaching methods and my own pedagogical
approach through listening to various podcasts, especially ``the teaching in
higher-ed podcast.'' I find it hard to see myself directly using the
methods discussed there or in the patch in my own lectures. I am particularly
interested in the idea of the flipped classroom~\cite{Abeysekera:2015}.
However, I am unsure how to encourage the students to prepare at home. If they
do not do the preparation for the lecture in the flipped classroom then they
will be lost during the in-class exercises.

I am not convinced that more control enhances learning or engagement.
Tests and quizzes do give a metric for if the students have learned the
basic concepts but I fell that rote memorization is at odds with
creativity. What I would really appreciate is a way to build measurement
of synthesizing knowledge into my lesson plan. Currently this
is done by recalling previous lecture's concepts during a lecture but I
feel that this does not help the students who are struggling to catch
up.

Teaching Computer Science in many ways is similar to teaching an
artistic discipline more than some other disciplines. For example, one
can memorize colour theory but it requires multiple practical
applications to have the student really understand the theory on a deep
level that then will propel the leaps of imagination. This is true in
computer science as well. For example, learning how to enter data into
and out of a database is an important skill that students can use to do
a research project or get a job later. However, technology is constantly
changing. If students just memorize the current database technology or
programming language then the world will pass them by in a few years.
The theory behind these technologies does not change as fast, however.
Understanding the (usually mathematical) theory behind the technologies
we use makes students more creative and adaptable.

This concept was also difficult to deliver during my microteaching
exercise. It was not clear to me how to distill a 1.5 hour class where I
wanted feedback on how to better tie theory and practice together. Five
minutes is enough time to get feedback on presentation style but not how
the structure contributes to learning. I think it would have been better
to start with a simpler concept, maybe from an earlier year, which would
show the two pieces (math and implementation).

It seems to me, though, that the ideas expressed in this patch are
structured around a model of factual learning and categorization of
information. They also seem to assume a large time investment for
preparing lectures and modules. The lesson planning and presentation
structure recommendations given in the patch reflect this idea. For
example, preparing a lecture from scratch rather than using one from the
previous coordinator would take almost a week between reading the source
material, understanding the important points, developing examples, and
structuring the lecture. This amount of time investment is simply not
available to me at the moment. I would really have appreciated some
advice on quickly adapting previous lectures.

Furthermore, a lot of attention is put on The idea of the beginning, middle,
and end of the lecture seems to be at odds with presentation style
recommendations~\cite{Reynolds:2007,Duarte:2008}. For example, students will be
paying the most attention at the beginning of the lecture so why not teach the
important summary concepts at the beginning of the lecture and then go into
less and less important things as the lecture goes on? I have heard this
referred to as ``newspaper style'' since most people only read the first few
paragraphs of a newspaper article, the most important information is put first.
Lectures are not in general in this format and some tips on how to adapt old
lectures quickly to my own style would have been very helpful.

\hypertarget{now-what}{%
\section{Now what?}\label{now-what}}

I believe that effective learning is to both be oneself as an educator
and to unlock the student's individual potential and understanding of
the subject. It would be valuable to continue to explore various
teaching methods as not every student can be fit into a box. I enjoy
teaching and want to find additional resources to better serve my
students, my department and my university.

Going forward, I do intend to try and incorporate some more quizzes into my
lectures. I want to look up some advice on how to test for knowledge synthesis
rather than memorization. My first place to look will be the e-learning
community. There are also some examples of how to measure understanding within
the visualization community~\cite{Sedig:2003,Sajaniemi:2004}.

I also have a current reading list to go through. My hope is that these
books will have some practical tips to address preparing lectures
quickly and how to link theory to practice. The main book in this set is
``The education of a graphic designer''~\cite{Heller:2005}.
Since graphic
design also incorporates the ideas of integrating theory and practice I
hope that the lessons in the book will be transferable to my field. I
also want to understand more about the goal of what is learning so I can
better design my lectures. Current books in this direction are ``How
humans learn''~\cite{Tyler:2018} and ``Small teaching''~\cite{Lang:2016}.

My hope is that these different sources will give me insight into how I can
better plan and deliver my lectures. The microteaching idea itself is a good
one. Maybe I can try to plan small examples and deliver them to get rapid
feedback~\cite{McNall:2004}. Incorporating both theoretical ideas from books
such as ``How humans learn'' with practical ideas from ``The education of a
graphical designer'' will help me prepare and deliver the lectures in an
engaging way while understand the reasons behind my decision.

\bibliographystyle{apacite}
\bibliography{writeup}

\end{document}


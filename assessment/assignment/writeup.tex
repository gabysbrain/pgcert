% Options for packages loaded elsewhere
\PassOptionsToPackage{unicode}{hyperref}
\PassOptionsToPackage{hyphens}{url}
%
\documentclass[
]{article}
\usepackage{lmodern}
\usepackage{amssymb,amsmath}
\usepackage{ifxetex,ifluatex}
\ifnum 0\ifxetex 1\fi\ifluatex 1\fi=0 % if pdftex
  \usepackage[T1]{fontenc}
  \usepackage[utf8]{inputenc}
  \usepackage{textcomp} % provide euro and other symbols
\else % if luatex or xetex
  \usepackage{unicode-math}
  \defaultfontfeatures{Scale=MatchLowercase}
  \defaultfontfeatures[\rmfamily]{Ligatures=TeX,Scale=1}
\fi
% Use upquote if available, for straight quotes in verbatim environments
\IfFileExists{upquote.sty}{\usepackage{upquote}}{}
\IfFileExists{microtype.sty}{% use microtype if available
  \usepackage[]{microtype}
  \UseMicrotypeSet[protrusion]{basicmath} % disable protrusion for tt fonts
}{}
\makeatletter
\@ifundefined{KOMAClassName}{% if non-KOMA class
  \IfFileExists{parskip.sty}{%
    \usepackage{parskip}
  }{% else
    \setlength{\parindent}{0pt}
    \setlength{\parskip}{6pt plus 2pt minus 1pt}}
}{% if KOMA class
  \KOMAoptions{parskip=half}}
\makeatother
\usepackage{xcolor}
\IfFileExists{xurl.sty}{\usepackage{xurl}}{} % add URL line breaks if available
\IfFileExists{bookmark.sty}{\usepackage{bookmark}}{\usepackage{hyperref}}
\hypersetup{
  pdfauthor={Thomas Torsney-Weir},
  hidelinks,
  pdfcreator={LaTeX via pandoc}}
\urlstyle{same} % disable monospaced font for URLs
\setlength{\emergencystretch}{3em} % prevent overfull lines
\providecommand{\tightlist}{%
  \setlength{\itemsep}{0pt}\setlength{\parskip}{0pt}}
\setcounter{secnumdepth}{-\maxdimen} % remove section numbering

\usepackage{apacite}

\title{Reflections on portfolio assessment}
\author{Thomas Torsney-Weir (849707)}
\date{}

\begin{document}
\maketitle

I try to encourage students in the modules I teach to be creative and I
want students to understand both theory and practice of the subject so
that they will be more ``future proofed'' to changes in technology. To
accomplish this I want to integrate a portfolio-style
assessment~\cite{koretz:1994} into my modules. I think the project portfolio
assessment method will make an interesting next step in making better
assignments. Students will be encouraged to be creative while
demonstrating that they've learned the concepts for the assignment.

To reflect on my ideas with regards to assessment methods, I will be using
Kolb's experiential learning model~\cite{kolb:2014}. I used this model because
I have been evaluating assessment methods for a while now. The cyclical nature
of Kolb's model fits better with a series of reflection and action on a
particular topic. Single pass models like that of Borton's model of
reflection~\cite{borton:1970} are more suited for more specific reflection or
newer ideas.

In order to understand the reasoning and personal issues with some other
methods I have tried, I will first talk about my experiences with other
assessment methods. This will give context to my choice of portfolio
assessment methods.

The first \emph{concrete experience} with assessment began with a
visualization module that I was teaching. One coursework for the module
was an individual programming exercise where the students were given a
specification (i.e.~a list of features) to implement. The assignment
also included a rubric describing how many points each feature is worth.

After a few times giving the same assignment, I became more and more
unhappy with the submissions. The students were accomplishing the
assignment but not much learning seemed to be taking place. Upon
\emph{reflective observation} of the situation I noticed a great deal of
similarity between the student assignments. With few exceptions,
students could replicate the instructions in the tutorial but not go any
further. This was confirmed by talking to the students. They were mostly
copying from online tutorials without trying to understand the copied
code. Furthermore, they weren't synthesizing the information. I was
quite uncomfortable with the assignment as is. Furthermore, the
assignment was quite boring to mark since all assignments were so
similar. What I wanted was some way to encourage creativity while
accomplishing the assignment.

Before changing the assignment I considered what I want students to
learn in my modules. This is the \emph{abstract conceptualization} step
of Kolb's model. I reflected on what I felt are important skills for
students to take into the real world. I based this on both the
pedagogical goals of computer science and my own experiences from
working in industry. Computer science is an applied field. Much of
computer science education is teaching current programming languages and
techniques that can be directly applied in industry. However, there is
also a large theoretical component with the idea that this will make
students more adaptable to inevitable changes in technologies~\cite{lethbridge:2000}.
In industry soft skills and developing creative solutions to problems
are also vital.

Therefore, there are three aspects to every module that I want students
to learn: ability to explain their work, showing creative freedom, and
synthesizing concepts in the module. Being able to communicate their
contributions, both how and why they are important, is a vital skill in
the job market. Creativity is also a much-valued skill in the technology
world and that should be encouraged within modules as well. Creativity
should not be overly constrained by the specification in the assignment.
Finally, I want the students to demonstrate that they are not only able
to repeat concepts in the module but that they have internalized
concepts. This is important because technologies change rapidly. A
popular programming language or framework today may not be in active use
in five years time. The theory and concepts do not change as fast so a
firm understanding of these concepts will help students to be more
adaptable to changes in the future.

These goals match with the `synthesize' aspect of Bloom's
taxonomy~\cite{bloom:1956}. Learning objectives were easy to specify what I
wanted students to demonstrate. There was a disconnect between the ways
of accomplishing the specification versus what the students were
learning.

Based on this, I created an experiment for the next time I taught the
module (\emph{active experimentation}). I changed the assignment to give
a general problem description and dataset along with a new rubric for
the main programming assignment. Rather than using features as the basis
of evaluation, I developed a hierarchy of learning objective-based
marks. Students could accomplish the assignment however they saw fit and
just had to demonstrate that they had learned the concepts for the
assignment. The idea of this type of assignment was to encourage
creativity in accomplishing the assignment. My hope was also to avoid
submissions where code was copy-pasted from tutorials without any
understanding.

In order to get the students started I provided a sample image of the
final tool and some guidelines on how to start. However, I reiterated in
the assignment that students could accomplish it however they liked. I
also asked for feedback on this new type of assignment.

While students worked on the assignment I received many questions about
how they should get started. The questions centered around the fact that
the assignment and rubric did not give a clear specification for the
assignment. These did not happen in the original version of the
assignment. Furthermore, the difficulty of the open-ended assignment
resulted in students simply implementing the sample version.

I used another round of \emph{reflective observation} of the assignment,
this time using the new version, to figure out the source of student
difficulty. It was clear that the difficulty of the assignment wording
was discouraging creativity. Furthermore, it was not clear that the
students were internalizing the concepts.

My \emph{abstract conceptualization} of the problem is that students are
not familiar with more open-ended assignments. This is understandable.
In other modules, students are given a fairly rigid specification with a
defined list of features. Assessment is based on how many of these
features are present and functional. The previous version of the
assignment also did not develop any soft skills. For my next revision of
this assignment I will take into account the student expectations while
trying to accomplish my goals.

My idea going forward is moving to using portfolios method in
my modules. Portfolios seem to have a number of advantages over a single
assignment. They encourage individuality. Each student is encouraged to
develop a distinctive portfolio. Students can use the portfolio as part
of their job interview process. Many technology companies want to see
samples of what sorts of projects applicants have done and how they
accomplished them. Portfolios provide these samples. They also allow the
students to show individual work, allowing students to distinguish
themselves and encourages creative freedom.

I will split up a coursework into several smaller assessments which
would form a portfolio. Perhaps I can integrate the coursework with the
practical lab sessions in the module. By breaking up the assignment into
smaller parts I hope to encourage experimentation. The first smaller
assignment can also be used to get the students used to the different
specification and rubric. In addition the portfolio will give the
students an opportunity to get early feedback. From a lecturing
standpoint, this will give me the chance to see if students are falling
behind. Then, steps can be taken to catch up the student. Furthermore,
the portfolio structure allows for rewarding of extra work. This can
either be done for students who are falling behind or for exceptional
students to progress further. I can also adjust the mark based on
improvement during the semester. Finally, a reflective assignment will
help the students to develop soft skills of explaining what they've done
and why.

One issue with this assessment method is that it might encourage too
much individualism. Learning how to work in a group is an important
skill but the portfolio method seems to encourage individualism. A vital
skill in the technology world is learning to integrate one's own
individual ideas with others in the group. I also worry that students
who are better at ``marketing'' their contributions may trump technical
ability. While soft skills are important they should not a replacement
for technical skill.

The biggest issue with portfolio assessment is that evaluation is quite
expensive in terms of time. Each assignment will be unique so scaling this
assessment method to fifty or 100 students may not be possible.  Typical
solutions to having many students is to simplify the assessment (e.g.~multiple
choice questions) or use an automatic code checking system. The problem with
these is they require a rigid assignment specification which does not allow
creative solutions. Two solutions I have been considering is using peer
marking~\cite{gielen:2010,fry:1990}. With peer marking it is not clear to me of
the quality of marking especially in terms of feedback with more open-ended
assignments. My other idea is to have a short question and answer section and
mark mostly on that but then technical ability will be de-emphasized. Without a
clear way to handle this it will be difficult to create the next experiment and
see how well the portfolio assessment method works in practice in my modules.

\bibliographystyle{apacite}
\bibliography{writeup}

\end{document}

